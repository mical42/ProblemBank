\documentclass[12pt]{exam}
\usepackage[utf8]{inputenc}
\usepackage[margin=1in]{geometry}
\usepackage{amsmath,amssymb,verbatim}
\printanswers
\newcommand{\limi}[2]{\lim\limits_{{#1}\rightarrow{#2}}}
\begin{document}
\pagestyle{headandfoot}
\firstpageheadrule
\firstpageheader{Math 221}{Worksheet}{September 18, 2018}
\runningheader{}{}{}
\firstpagefooter{}{}{}
\runningfooter{}{}{}

	\begin{questions}
		\question Define $h(x)=\left\{\begin{array}{cl}
        x+3, & x<-1\\
        -x+3, & -1\leq x<1, \textrm{ or } 1<x<2\\
        4, & x=1\\
        x-1, & x\geq2\\
        \end{array}\right.$ 
        
        Sketch the function, and evaluate the following.
        \begin{parts}
		\part $\limi{x}{-1^-}h(x)$, $\limi{x}{-1^+}h(x)$, $\limi{x}{-1}h(x)$, and $h(-1)$
		\begin{solution}[4cm]
			    $2,4,DNE,4$
		    \end{solution}
            \part $\limi{x}{1^-}h(x)$, $\limi{x}{1^+}h(x)$, $\limi{x}{1}h(x)$, and $h(1)$
		\begin{solution}[4cm]
			    $2,2,2,4$
		    \end{solution}
	\end{parts}

\question
        Suppose $f(x)$ and $g(x)$ are functions, and you know $\limi{x}{a}f(x)=L$, $\limi{x}{a}g(x)=K\neq0$. Evaluate the following limits in terms of $L$ and $K$.
        \begin{parts}
            \item $\limi{x}{a}\frac{f(x)}{g(x)}$
		    \begin{solution}[2.5cm]
			    $\frac{L}{K}$
		    \end{solution}
            \item $\limi{x}{a}\Big(f(x)\Big)^2$
		    \begin{solution}[2.5cm]
			    $L^2$
		    \end{solution}
            \item $\limi{x}{a}\left(\frac{f(x)-L}{g(x)+K}\right)$
		    \begin{solution}[2.5cm]
			    $0$
		    \end{solution}
        \end{parts}
    	
\question True or false:
        \begin{parts}
            \part If $\limi x5 f(x)=0$ and $\limi x5 g(x)=0$, then $\limi x5\frac{f(x)}{g(x)}$ does not exist.
		\begin{solution}[3.5 cm]
					F	
					\end{solution}
            \part If $f(x)>1$ for all $x$ and if $\limi x0 f(x)$ exists, then $\limi x0 f(x)>1$. 
		\begin{solution}[3.5 cm]
					F	
					\end{solution}
	\part True or false: If $\limi x6 f(x)g(x)$ exists, then the limit must be $f(6)g(6)$. 
		\begin{solution}[3.5cm]
						F
					\end{solution}
            \part If $\limi x0f(x)=\infty$ and $\limi x0g(x)=\infty$, then $\limi x0[f(x)-g(x)]=0$.
		\begin{solution}[3.5cm]
						F
					\end{solution}
            \part $\limi{x}{4}\left(\frac{2x}{x-4}-\frac8{x-4}\right)=\limi{x}4\frac{2x}{x-4}-\limi x4\frac8{x-4}$ 
		\begin{solution}[3.5cm]
						F
					\end{solution}
        \end{parts}

\question
        Evaluate the following limits, if they exist. If they exist, be ready to explain each step you take to evaluate the limits.
        \begin{parts}
		\part $\lim\limits_{x \to \frac{1}{2}^{-}} \frac{|2x-1|}{2x-1}$
		\begin{solution}[3cm]
			$-1$.
		\end{solution}
		\part $\lim\limits_{x \to \frac{1}{2}} \frac{|2x-1|}{2x - 1}$
		\begin{solution}[3cm]
			DNE
		\end{solution}
            \part $\lim\limits_{x \to 1}\frac{x^3-1}{x-1}$
		\begin{solution}[3cm]
\[\limi x1\frac{x^3-1}{x-1}=\limi x1\frac{(x-1)(x^2+x+1)}{x-1}=\limi x1 x^2+x+1=1+1+1=3   \]
	    \end{solution}
	    \part $\lim\limits_{x \to 4}\sqrt[3]{x^2-5x-4}$
		\begin{solution}[3cm]
 \[\limi x4\sqrt[3]{x^2-5x-4}=\sqrt[3]{4^2-5(4)-4}=\sqrt[3]{16-20-4}=\sqrt[3]{-8}=-2\]
	    \end{solution}
	    \part $\lim\limits_{h \to 0} \frac{(4+h)^2-16}h$
		\begin{solution}[3cm]
\[\limi h0 \frac{(4+h)^2-16}h=\limi h0 \frac{(16+8h+h^2)-16}{h}=\limi h0 \frac{8h+h^2}{h}=\limi h0 8+h=8 \]
	    \end{solution}
           % \part Find $\limi{x}{16}\frac{2\sqrt x+x^{3/2}}{\sqrt[4]{x}+5}$
           % \part $\limi h0\frac{3}{\sqrt{3h+1}+1}$
		\part $\lim\limits_{x \to-3}\frac{2-\sqrt{x^2-5}}{x+3}$ (Hint: multiply by the conjugate)
		\begin{solution}[3cm]
\[\limi x{-3}\frac{2-\sqrt{x^2-5}}{x+3}=\limi x{-3}\frac{2-\sqrt{x^2-5}}{x+3}\cdot\frac{2+\sqrt{x^2-5}}{2+\sqrt{x^2-5}}\]
            \[=\limi x{-3}\frac{4-(x^2-5)}{(x+3)(2+\sqrt{x^2-5})}=\limi x{-3}\frac{-x^2+9}{(x+3)(2+\sqrt{x^2-5})}\]
            \[=\limi x{-3}\frac{9-x^2}{(x+3)(2+\sqrt{x^2-5})}=\limi x{-3}\frac{(3-x)(3+x)}{(x+3)(2+\sqrt{x^2-5})}\]
            \[=\limi x{-3}\frac{3-x}{2+\sqrt{x^2-5}}=\frac{3-(-3)}{2+\sqrt{3^2-5}}=\frac6{2+\sqrt4}=\frac32
            \]
	    \end{solution}
           % \part Determine $\limi x9\frac{\sqrt x-3}{x-9}$
            \part  $\lim\limits_{x \to 7}\frac{\sqrt{x+2}-3}{x-7}$
		\begin{solution}[3cm]
\[\limi{x}{7}\frac{\sqrt{x+2}-3}{x-7}=\limi{x}{7}\frac{\sqrt{x+2}-3}{x-7}\cdot\frac{\sqrt{x+2}+3}{\sqrt{x+2}+3}=\limi{x}{7}\frac{(x+2)-9}{(x-7)(\sqrt{x+2}+3}\]
            \[=\limi{x}{7}\frac{x-7}{(x-7)(\sqrt{x+2}+3)}=\limi{x}{7}\frac{1}{\sqrt{x+2}+3}=\frac1{\sqrt{7+2}+3}=\frac16 \]
	    \end{solution}
		\begin{comment}
            \part $\lim\limits_{x \to -2}\frac{-2x-4}{x^3+2x^2}$
	    \begin{solution}
\[\limi x{-2}\frac{-2x-4}{x^3+2x^2}=\limi x{-2}\frac{-2(x+2)}{x^2(x+2)}=\limi x{-2}\frac{-2}{x^2}=\frac{-2}4=-\frac12 \]
	    \end{solution}
		\end{comment}
		\part $\lim\limits_{x \to 0} x^4 \cos (1/x)$ (Hint: use the squeeze theorem like we did in class.)
		\begin{solution}[3cm]
			$0$.
		\end{solution}
	\end{parts}
	\question Say $f(x) = 3x - 6$. Note $\lim\limits_{x \to 1} f(x)= -3$. Recall the distance between two numbers $a,b$ is given by $|b-a|$.
	\begin{parts}
		\part How close does $x$ need to be to $1$ so that $f(x)$ is at most $2$ away from $-3$ (when $x \neq 1$)?
		\begin{solution}[3cm]
			$\frac{2}{3}$.
		\end{solution}
		\part How close does $x$ need to be to $1$ so that $f(x)$ is at most $0.3$ away from $-3$ (when $x \neq 1$)?
		\begin{solution}[3cm]
			$0.1$.
		\end{solution}
		\part How close does $x$ need to be to $1$ so that $f(x)$ is at most $0.02$ away from $-3$ (when $x \neq 1$)?
		\begin{solution}[3cm]
			$\frac{0.02}{3}$.
		\end{solution}
		\part How close does $x$ need to be to $1$ so that $f(x)$ is at most $\epsilon$ away from $-3$ (when $x \neq 1$)? (Your answer should be in terms of $\epsilon$)
		\begin{solution}[3cm]
			$\epsilon/3$.
		\end{solution}	
		\part Find $\delta$, in terms of $\epsilon$, so that when $0 < |x-1|< \delta$, we have $|f(x) - (-3)| < \epsilon$.
\begin{solution}[3cm]
			$\epsilon/3$.
		\end{solution}
	\end{parts}
	\end{questions}
\end{document}
