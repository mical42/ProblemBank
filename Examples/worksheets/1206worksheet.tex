\documentclass[12pt]{exam}
\usepackage[utf8]{inputenc}
\usepackage[margin=1in]{geometry}
\usepackage{amsmath,amssymb,verbatim}
%\usepackage{graphicx}
\printanswers
\newcommand{\limi}[2]{\lim\limits_{{#1}\rightarrow{#2}}}
\begin{document}
\pagestyle{headandfoot}
\firstpageheadrule
\firstpageheader{Math 221}{Worksheet}{December 6, 2018}
\runningheader{}{}{}
\firstpagefooter{}{}{}
\runningfooter{}{}{}

	\begin{questions}
		\question Find the volume of a pyramid with square base of length $4$ and height $12$.
\begin{solution}[3cm]
\end{solution}


\question Consider the region bounded by $y=e^x,\;y=0,\;x=0,\;$ and $x=1$.
		\begin{parts}
			\part Draw the region above. Also label the line segment with endpoints $(x,0)$ and $(x,e^x)$ on your drawing.
			\begin{solution}[2cm]
			\end{solution}
			\part Suppose I rotate this line segment about the line $x=-1$. What shape will this rotated line segment generate? What surface area will it have? (Hint: to find the surface area of a cylinder, find the height and the radius of the cylinder, and ``unroll'' the cylinder into a sheet of paper.)
			\begin{solution}[3cm]
			\end{solution}
			\part Call the surface area you got above $A(x)$. Now consider the solid of revolution generated by rotating the above region about the axis $x=-1$. What is the volume of the solid, using the method of cylindrical shells? (Just set up the integral, you don't have to compute.)
		\begin{solution}[4cm]

\end{solution}

\part What if we rotate the region about the line $x=-2$?
\newpage
		\end{parts}
	
		\question Use the method of shells to set up the integral for the volume (don't compute!) of the solids obtained by rotating:

\begin{parts}
\part  the region bounded by $x=1,\;x=y,\;y=0$ around the line $x = -2$;
\begin{solution}[3cm]
			\end{solution}

\part  the region bounded by $x=y^2,\;y=\frac{x}{ 2}$ around the line $x=5$
\begin{solution}[3cm]
			\end{solution}

\end{parts}


		\question For the following, ({\it set up} the integrals which represent
the volume using the shell method {\it and} disc method (i.e. set up the
integrals two ways), if at all possible.  Decide which integral would be easier to calculate
(but don't do it!).

\begin{parts}
\part the region bounded by $y=\cos x + 1,\;y=0,\;x=2 \pi$, rotated around the $y$-axis.
	\begin{solution}[4cm]
	\end{solution}

\part  the region bounded by $y=\cos x,\;y=\sin x,\;x=0,\;x=\frac{\pi }{ 4}$
rotated about the x-axis;
\begin{solution}[4cm]
			\end{solution}
\end{parts}

		\question The Jamaican bobsled team is practicing their running start. They have 50 meters to accelerate their bobsled from a full stop to a speed of 11 meters per second (approximately 24 miles per hour). %Their bobsled weighs $250$ kg.
		However, as they run alongside the sled, it becomes harder to accelerate it: the team can only provide an force of $100-2x$ Newtons when they are $x$ meters along the runway. How much work will they do to their bobsled?
\begin{solution}[3cm]
\end{solution}

			\end{questions}
\end{document}
