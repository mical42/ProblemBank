\documentclass[12pt]{exam}
\usepackage[utf8]{inputenc}
\usepackage[margin=1in]{geometry}
\usepackage{amsmath,amssymb,verbatim}
\usepackage{graphicx}
\usepackage{array}
\newcolumntype{C}[1]{>{\centering\let\newline\\\arraybackslash\hspace{0pt}}m{#1}}

\printanswers
\newcommand{\limi}[2]{\lim\limits_{{#1}\rightarrow{#2}}}
\begin{document}
\pagestyle{headandfoot}
\firstpageheadrule
\firstpageheader{Math 221}{Worksheet}{October 18, 2018}
\runningheader{}{}{}
\firstpagefooter{}{}{}
\runningfooter{}{}{}

	\begin{questions}
		\question
         Suppose $f(x)$ is a differentiable function with $f(1)=10$, and $f'(x)\geq 2$ for $1\leq x\leq 4$. Use the mean value theorem to find the smallest possible value of $f(4)$.
		\begin{solution}[4cm]
        The MVT tells us there is a point $c$ in the interval $(1,4)$ such that 
        \[f'(c)=\frac{f(4)-f(1)}{4-1}\]
        Simplifying the right hand side, we get \[\frac{f(4)-f(1)}{4-1}=\frac{f(4)-10}{3}\]
        Now since we know that $f'(x)\geq2$ on this interval, in particlar we know that $f'(c)\geq2$. So we have \[2\leq f'(c)=\frac{f(4)-10}{3}\]
        Solving this inequality for $f(4)$ we have \[2\leq\frac{f(4)-10}{3}\] 
        \[6\leq f(4)-10\] \[16\leq f(4)\]
        So the smallest possible value of $f(4)$ is 16. 
        
        (You could also draw a picture to figure this out).
    \end{solution}
		\question Does there exist a differentiable function $g(x)$ such that $g(0)=-1$, $g(2)=4$, and $g'(x)\leq 2$ for all $x$? Find an example or explain why it doesn't exist.
		\begin{solution}[3.5cm]
        If such a function existed, then by MVT we would have some $c$ in the interval $(0,2)$ such that \[g'(c)=\frac{g(2)-g(0)}{2-0}=\frac{4-(-1)}2=\frac52\]
        This is impossible since $g'(x)\leq2$ for all $x$.
    \end{solution}

        \question Determine whether $f(x)$ satisfies the hypothesis of the mean value theorem on the given interval. If so, find all numbers $c$ in the interval so that $f(b)-f(a)=f'(c)(b-a)$.
        \begin{parts}
        \part $f(x)=\frac{1}{(x-1)^2}$, $[0,2]$
		\begin{solution}[3.5cm]
No, $f$ is not continuous on $[0,2]$
		\end{solution}
        \part $f(x)=x+\frac4x$, $[1,4]$
		\begin{solution}[3.5cm]
$c=2$

		\end{solution}
        \part $f(x)=4+\sqrt{x-1}$, $[1,5]$
		\begin{solution}[3.5cm]
$c=2$
		\end{solution}
  %      \part $f(x)=\sin x$, $[0,\frac{\pi}2]$
%		\begin{solution}[3cm]
%$c=\arccos\frac2{\pi}$
%		\end{solution}
	\end{parts}
        
	\question Explain why the Mean Value Theorem applies to the following statements.
        \begin{parts}
        \part If two runners start a race at the same time and finish at the same time, there be some point in the race where they have the same speed.
		\begin{solution}[3cm]
		\end{solution}
        \part It is impossible to travel 50 miles in an hour without ever going 50 mph.
\begin{solution}[3cm]
		\end{solution}
        \end{parts}

	\end{questions}
\end{document}
