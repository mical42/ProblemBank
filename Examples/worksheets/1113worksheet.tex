\documentclass[12pt]{exam}
\usepackage[utf8]{inputenc}
\usepackage[margin=1in]{geometry}
\usepackage{amsmath,amssymb,verbatim}
%\usepackage{graphicx}
\printanswers
\newcommand{\limi}[2]{\lim\limits_{{#1}\rightarrow{#2}}}
\begin{document}
\pagestyle{headandfoot}
\firstpageheadrule
\firstpageheader{Math 221}{Worksheet}{November 13, 2018}
\runningheader{}{}{}
\firstpagefooter{}{}{}
\runningfooter{}{}{}

	\begin{questions}
		\question Evaluate the following definite integrals.
	\begin{parts}
%		\part $\int\limits_{-10}^{10} x e^{x^4 - 3} dx$. (Hint: Notice this function is odd)
%		\part $\int\limits_{0}^{1} \frac{1}{1+x^2} dx$. 
%		\part $\int\limits_{-3}^3 x \cos (x^2 + 1) dx$.
		\part $\int\limits_{-3}^{12} f(x) dx$ given $\int\limits_{-3}^4 f(x) dx = 5$, $\int\limits_{2}^{12} f(x) dx = -3$ and $\int\limits_{2}^4 f(x) dx = 6$.
\begin{solution}[2cm]
	$\int\limits_{-3}^{12} f(x) dx = \int\limits_{-3}^{4} f(x) dx  + \int\limits_{4}^{12} f(x) dx = \int\limits_{-3}^{4} f(x) dx  + \int\limits_{2}^{12} f(x) dx - \int\limits_{2}^{4} f(x) dx = 5 + (-3) - 6 = -4$.
	\end{solution}
		\part $\int\limits_{-7}^0 f(x) dx$ given $f(x)$ is odd, i.e. $f(-x) = - f(x)$, and $\int\limits_{0}^7 f(x) dx = -4$.
\begin{solution}[2cm]
	Since $f(x)$ is odd, we have $\int\limits_{-7}^0 f(x) dx = - \int\limits_{0}^7 f(x) dx = 4$.
	\end{solution}
		\part $\int\limits_{-13}^{12} f(x) dx$ given $f(x)$ is odd and $\int\limits_{-13}^{-12} f(x) = 4$.
\begin{solution}[2cm]
	Write $\int\limits_{-13}^{12} f(x) dx = \int\limits_{-13}^{-12} f(x) dx + \int\limits_{-12}^{12} f(x) dx = 4 + 0 = 4$
	\end{solution}
		\part $\int\limits_{-2}^{8} f(x) dx$ given $f(x)$ is even (meaning $f(-x) = f(x)$) and $\int\limits_{0}^2 f(x) dx = 7$ and $\int\limits_{2}^{8} f(x) dx = -3$.
\begin{solution}[2cm]
$\int\limits_{-2}^{8} f(x) dx = \int\limits_{-2}^{0} f(x) dx + \int\limits_{0}^{2} f(x) dx+ \int\limits_{2}^{8} f(x) dx = \int\limits_{0}^{2} f(x) dx + \int\limits_{0}^{2} f(x) dx+ \int\limits_{2}^{8} f(x) dx = 7 + 7 + (-3) = 11$
	\end{solution}
		\part $\int\limits_{-14}^5 f(x) dx$ given $f(x)$ is even, $\int\limits_{-14}^{-5} f(x) = 12$ and $\int\limits_{0}^5 f(x) dx = -9$.
\begin{solution}[2cm]
$\int\limits_{-14}^5 f(x) dx = \int\limits_{-14}^{-5} f(x) + \int\limits_{-5}^0 f(x) dx+ \int\limits_{0}^5 f(x) dx = 12 + (-9) + (-9) = -6$.
	\end{solution}
	\end{parts}

		\question Suppose I am walking in a straight line across the frozen lake. My shoes are very slick, and the winds are very strong, so although I try to advance forward, often the wind blows me back as I slip across the ice. My velocity is given by $v(t) = (t-3)(t-1)$ miles per hour, where $t$ is in hours
		\begin{parts}
			\part Suppose my initial position is $1$ mile from shore. How far away am I from shore after $4$ hours?
			\begin{solution}[3cm]
			\end{solution}
			\part Find the total distance I have traveled after 4 hours (recall the total distance traveled is the integral of the \textit{speed}).
\begin{solution}[3cm]
			\end{solution}
			\part The average value of a function $f$ over the interval $[a,b]$ is defined as $\frac{1}{b-a} \int\limits_{a} ^b f(t) dt$. What is my average velocity over the first $4$ hours? What is my average speed over the first $4$ hours?
			\begin{solution}[3cm]
			\end{solution}

		\end{parts}

		\question Compute the following integrals (you may have to use $u$-substitution).
		\begin{parts}
			\part $\int x^5 \sin (x^6 + 9) dx$.
\begin{solution}[3cm]
	Set $u = x^6 + 9$, so that $\frac{du}{dx} = 6 x^5$, or $du = 6x^5 dx$. Thus $\int \sin (x^6 + 9) x^5 dx = \int \sin u \frac{du}{5} = \frac{-\cos(u)}{5} + C = \frac{-\cos (x^6 + 9}{5} + C$.
			\end{solution}
			\part $\int 12 x (x+3)^2 dx$.
\begin{solution}[3cm]
	$u = x+3$. Then $du = dx$. Then $\int 12 x (x+3)^2 dx = \int 12 (u-3) (u)^2 du =  \int 12 u^3 - 12 \cdot 3u^2 du = 3 u^4 - 12 u^3 + C = 3 (x+3)^4 - 12 (x+3)^3 + C$.
			\end{solution}
			\part $\int 90 t (t-7)^7 dt$.
\begin{solution}[3cm]
	$u = t-7$, so $du = dt$. We get $\int 90 t (t-7)^7 dt = \int 90 (u+7) (u)^7 du = \int 90 u^8 + 7 \cdot 90 u^7 du = 10 u^9 + \frac{7 \cdot 90}{8} u^8 + C = 10 (t-7)^9 + \frac{7 \cdot 90}{8} (t-7)^8$.
			\end{solution}
			\part $\int \sqrt{12y + 6} dy$.
\begin{solution}[3cm]
	$u= 12y + 6$, so $du = 12 dy$. Then we get $\int \sqrt{u} \frac{du}{12} = \frac{1}{12} \int u^{1/2} du = \frac{1}{12} \frac{u^{3/2}}{3/2} + C = \frac{u^{3/2}}{18}$.
			\end{solution}
			\part $\int y^{8} \sqrt{4 y^3 - 6} dy$. (Hint: try expressing your leftover $y$'s in terms of your choice of $u$).
\begin{solution}[3cm]
	$u = 4y^3 - 6$, so $du = 12 y^2 dy$. $\int \sqrt{4 y^3 -6} y^8 dy = \int \sqrt{u} y^6 \frac{du}{12} = \int \frac{\sqrt{u} (y^3)^2}{12} du = \int \frac{u^{1/2} (\frac{u+6}{4})^2 du = \int \frac{1}{16} u^{1/2} (u^2 + 12 u + 36) du = \frac{1}{16} [\frac{u^{7/2}}{7/2} + 12 \frac{u^{5/2}}{5/2} + 36 \frac{u^{3/2}}{3/2}] + C$
			\end{solution}
			\part $\int \frac{7x - 1}{7x^2 -2x + 3} dx$.
\begin{solution}[3cm]
			\end{solution}
			\part $\int \cos x \sqrt{ \sin x} dx$.
\begin{solution}[3cm]
	
			\end{solution}
			\part $\int \cos^3 x dx$ (Hint: $\cos ^2 x = 1 - \sin ^2 x$).
\begin{solution}[3cm]
			\end{solution}
			\part $\int\limits_0 ^{\pi} \frac{\sin x }{\cos ^2 x} dx$.
\begin{solution}[3cm]
			\end{solution}
			\part $\int\limits_{0}^1 \frac{x}{\sqrt{2-x^2}} dx$.
\begin{solution}[3cm]
			\end{solution}
			\part $\int\limits_{0}^2 (x-1)^{99} x dx$.
\begin{solution}[3cm]
			\end{solution}
			\part $\int\limits_{1}^4 \frac{\sqrt{2 + \sqrt{x}}}{\sqrt{x}} dx$.
\begin{solution}[3cm]
			\end{solution}
		\end{parts}
	\end{questions}
\end{document}
